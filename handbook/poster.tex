% The document class supplies options to control rendering of some standard
% features in the result.  The goal is for uniform style, so some attention 
% to detail is *vital* with all fields.  Each field (i.e., text inside the
% curly braces below, so the MEng text inside {MEng} for instance) should 
% take into account the following:
%
% - author name       should be formatted as "FirstName LastName"
%   (not "Initial LastName" for example),
% - supervisor name   should be formatted as "Title FirstName LastName"
%   (where Title is "Dr." or "Prof." for example),
% - degree programme  should be "BSc", "MEng", "MSci", "MSc" or "PhD",
% - dissertation title should be correctly capitalised (plus you can have
%   an optional sub-title if appropriate, or leave this field blank),
% - dissertation type should be formatted as one of the following:
%   * for the MEng degree programme either "enterprise" or "research" to
%     reflect the stream,
%   * for the MSc  degree programme "$X/Y/Z$" for a project deemed to be
%     X%, Y% and Z% of type I, II and III.
% - year              should be formatted as a 4-digit year of submission
%   (so 2014 rather than the accademic year, say 2013/14 say).
%
% Note there is a *strict* requirement for the poster to be in portrait 
% format so that we display them on the poster boards available.

\documentclass[ % the name of the author
                    author={Daniel Page},
                % the name of the supervisor
                supervisor={Dr. Andrew Calway},
                % the degree programme
                    degree={MEng},
                % the dissertation    title (which cannot be blank)
                     title={Some Structural Guidelines for CS MEng Posters},
                % the dissertation subtitle (which can    be blank)
                  subtitle={},
                % the dissertation     type
                      type={enterprise},
                % the year of submission
                      year={2014} ]{poster}

\begin{document}

% -----------------------------------------------------------------------------

\begin{frame}{} 

\vfill

\begin{columns}[t]
  \begin{column}{0.900\linewidth}
  \begin{block}{\Large Introduction}
  It is hard to give generic advice about what form your poster should 
  take, since each project relates to a different topic and each student
  will be at a different stage wrt. completeness.  Therefore, the best 
  approach is to focus on the underlying aim of the poster presentation:
  essentially the intention is for you to get early, objective opinions
  about your work and then (ideally) improve it as a result.

  With this in mind, one idea is to

  \begin{enumerate}
  \item think about how to explain your project to someone, and questions
        you might want an answer to or opinion on,
  \item consider the poster as a set of slides, which support an elevator
        pitch\footnote{\url{http://en.wikipedia.org/wiki/Elevator_pitch}}
        for either the technical and/or business plan part,
        then
  \item focus the poster content on the part you feel you need the most
        input on.
  \end{enumerate}

  \noindent
  Another approach is to adopt standard advice about developing research 
  posters\footnote{\url{http://www.ncsu.edu/project/posters/NewSite/}}, 
  then produce a stand-alone result that summarises your project (see
  examples on walls throughout the MVB).  Either way, the blocks below 
  attempt to outline some potential examples of content.
  \end{block}
  \end{column}
\end{columns}

\vfill

\begin{columns}[t]
  \begin{column}{0.422\linewidth}
  \begin{block}{\Large 1. Project Outline}
  Example content could follow initial specification, and might include:

  \begin{itemize}
  \item an outline of the problem context,
  \item a description of the central challenge, 
  \item an overview of the direction (within the possible options) you 
        have opted to take,
        and
  \item a concrete list of aims and objectives.
  \end{itemize}
  \end{block}
  \end{column}

  \begin{column}{0.422\linewidth}
  \begin{block}{\Large 2. Business Plan/Research Proposal}
  Example content might include:

  \begin{itemize}
  \item identification and analysis of a market,
  \item proposed product/service portfolio,
  \item ideas about development and protection of IP,
  \item proposed company organisation,
        and
  \item estimates for start-up and recurrent costs.
  \end{itemize}
  \end{block}
  \end{column}
\end{columns}

\vfill

\begin{columns}[t]
  \begin{column}{0.422\linewidth}
  \begin{block}{\Large 3. Preliminary Results}
  \vspace{10cm}
  \end{block}
  \end{column}
  \begin{column}{0.422\linewidth}
  \begin{block}{\Large 4. Progress and Status}
  Example content might include:

  \begin{itemize}
  \item a list of complete and incomplete aims and objectives,
  \item a list of open questions or problems,
        and
  \item your plan for completing the project, inc. required deliverables.
  \end{itemize}
  \end{block}
  \end{column}
\end{columns}

\vfill

\end{frame}

% -----------------------------------------------------------------------------

\end{document}



